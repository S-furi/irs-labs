%!TEX root = ../main.tex
\section{Lab Activity 2}
The task goal is a composite behaviour from the behaviours developed in the
previous lab (see \Cref{sec:lab01}). Some differences can be highlighted in
terms of constraints: here, the robot:
\begin{itemize}
    \item must reach the light as fast as possibile;
    \item once reached the light, it should stay close to it;
    \item wheel velocities cannot exceed $15^{-2}m/s$.
\end{itemize}
For the second constraint, the implementation provided in the first lab
activity already satisfies it. The robot goes towards the light, and when
passing it, the highest intensity will be in back sensors, causing the robot to
subsequently turn 180 degrees, repeating this process infinitely. When it comes
to the third constraint, wheel velocities had already an upper bound in the
previous implementation. Lastly, in order to satisfy the first constraint we
need first to find a way to integrate the collision avoidance behaviour with
phototaxis.

\subsection{Behaviours Integration} In order to integrate behaviours, a sort of
``prioritisation'' is needed, both for fusing and arbitration techniques. In
order to keep the controller as simple as possibile, an arbitration behaviour
coordination schema is implemented. In this case, at each step a behaviour
among all possibile behaviours is selected. In this particular task, we
prioritise collision avoidance with respect to phototaxis. We firstly check the
presence of obstacles and if some are found, we steer the robot away from the
obstacles. Otherwise, we keep following the light. In this way light reaching
can be slowed down by the collision avoidance task, but for real world
scenarios high prioritisation collision avoidance could be strongly desirable.

\subsection{Behaviours Design} It is way more easier to design behaviours
independently and separately instead of creating a composite behaviour from
scratch. Even if composing behaviour could be challenging, we have various ways
to archive this, for example by means of the subabsumption architecture,
cooperative schemes, competitive schemes, etc. In this task's case in
particular, having already defined behaviours in the previous lab activity made
this one more focused on how to integrate them archive the desired behaviour
with the given constraints.

\subsubsection{Collision Avoidance}
When it comes to the collision avoidance task, the same implementation is
provided as the one for the first lab activity. Briefly: the robot retrieves
the maximum proximity sensor's value through a neighbour-weighted average. If
the maximum value is higher that a threshold (0.55), based on the angle of the
sensor the robot steers left or right.

When it comes to behaviours coordination, this behaviour could benefit a memory
mechanism. In order to be more efficient for accomplishing task goals, instead
of going the opposite way of the obstacle, the robot could choose a free
direction that favours the task of phototaxis. This is especially true for all
cases where there is the presence of tall obstacles that blocks the light
source. If we can keep a reference to where the light was lastly located, we
could prefer directions that helps the robot going towards the light faster
when avoiding obstacles (or circumnavigating them).

\subsubsection{Phototaxis}
As per the collision avoidance behaviour, the phototaxis one is implemented as
the previous lab activity but with some key difference in order to make the
robot reach the light in the fastest way as possible, and in order to reduce
the chances of collision when steering towards the light. Firstly, when
detecting the light in front (i.e. max light value is detected from sensors 1
or the $24^{th}$), we set wheel velocities to max speed in order to reach light
faster. On the other hand, when light is detected on every other sensors, we
perform a left or right steering keeping the opposite wheel velocity to the max
value and the other to zero, reducing the turn radius with respect to the first
lab activity implementation. In this way, we can reduce possibile impacts with
obstacles when performing a turn in order to go towards the light.


\subsection{Tests}
The overall behaviour performs well even under challenging arena/robot
configuration, for example having a noise of 0.05 on senors, having various
kind of obstacles inside the arena, especially tall obstacles that blocks
light. Other tests also involves more that one robot. In this case, robots
successfully avoid collisions with themselves, and wander under the light once
reaching it, while avoiding collision with themselves.

\begin{table}[ht]
\centering
\begin{tabular}{c|cc|cc|cc|cc}
\textbf{Run} & \multicolumn{2}{c|}{\textbf{Arena 1} (391787)} & \multicolumn{2}{c|}{\textbf{Arena 2} (46753)} & \multicolumn{2}{c|}{\textbf{Arena 3} (833139)} & \multicolumn{2}{c}{\textbf{Arena 4} (406338)} \\
 & Score & Seed & Score & Seed & Score & Seed & Score & Seed \\
\hline
1 & 769 & 712659 & 817 & 525799 & 318 & 991166 & 564 & 733309 \\
2 & 656 & 856089 & 774 & 579969 & 345 & 418680 & 496 & 033018 \\
3 & 723 & 305032 & \textcolor{blue}{0.893} & 715223 & 321 & 483080 & \textcolor{blue}{2.094} & 986296 \\
4 & 468 & 830554 & \textcolor{blue}{0.870} & 986128 & 296 & 525029 & 943 & 185796 \\
5 & 310 & 227437 & \textcolor{blue}{1.653} & 237437 & 314 & 227003 & \textcolor{blue}{1.047} & 461587 \\
\end{tabular}
\caption{Simulation results on 1000 steps runs: integer scores indicate then
    number of steps in order to reach the light; floating-point scores (in
    \textcolor{blue}{blue}) indicate that the footbot could not reach the light
    in 1000 steps and records the minimum euclidean distance reached from the
    light. In column header, inside parenthesis are contained random seeds
    (argos) in order to produce the arena and robot placement. The seed column for
    each run determines the random seed used in the control software (re-generated
    at each reset).}
\label{tab:simulation-results}
\end{table}

In order to asses behaviour's quality, four experiments have been performed.
For each experiment a sufficiently complex arena has been produced which
contains six high walls (which block light) and a light placed in $(-1,0,0.5)$.
For each random arena generated, five runs with different robot's random seed
are performed, recording for each run the obtained \textbf{score}
(\Cref{tab:simulation-results}), which is defined as follows: A simulation
lasts 1000 time ticks. An integer is yielded if the robot could successfully
reach the light. The integer represents the $i$-th step in which the robot
reached the light. If the robot could not reach the light in 1000 steps, a
floating point number is returned, indicating the minimum euclidean distance
that the footbot has reached from the light in the whole run (so lower is
better). The arena and simulation parameters (even noise) are left in the
\texttt{lab02task1.argos} file.

