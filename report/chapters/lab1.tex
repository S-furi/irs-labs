%!TEX root = ../main.tex
\section{Lab Activity 1}

First lab activity has been about phototaxis and an implementation of a
collision avoidance behaviour with a simple random walk.
As discussed in class, phototaxis has been more challenging because
it has been our first approach in programming robots in a more structured
way, and also the first time interacting with ARGoS.

\subsection{Phototaxis}

Having no constraints on how fast the robot has to reach the light, and
no constraints in what the robot must do when reaching the light, designing
phototaxis is quite straightforward. Moreover, the arena is just composed
of its four walls, meaning that light cannot be obstructed by obstacles.

Having defined the environment and the constraints, the robot is designed
with a simple id

\begin{figure}[ht]
    \centering
    \begin{tikzpicture}
        \node[state, initial] (init) {init};
        \node[state, right=of init] (search) {Search Light};
        \node[state, above right=of search] (forward) {Move Forward};
        \node[state, below right=of search] (turn) {Turn};
        \node[state, right=of turn] (right) {Turn Right};
        \node[state, right=of forward] (left) {Turn Left};

        % Define transitions
        \draw[->] (init) -- node[midway, above] {Random Velocities} (search);
        \draw[->] (search) -- node[midway, above] {Light Directly Ahead} (forward);
        \draw[->] (search) -- node[pos=0.3, midway, below] {Light Not Directly Ahead} (turn);
        \draw[->] (turn) -- node[midway, below] {Angle $\geq$ 0} (right);
        \draw[->] (turn) -- node[midway, above] {Angle $<$ 0} (left);

        \draw[->] (right) .. controls +(up:2) and +(right:2) .. 
            node[pos=0.2, right] {Adjust Velocities} (search);
        \draw[->] (left) .. controls +(up:2) and +(right:2) .. 
            node[pos=0.1, left] {Adjust Velocities} (search);
        \draw[->] (forward) .. controls +(up:3) and +(up:3) .. 
            node[pos=0.2, left] {Set Max Speed} (search);
    \end{tikzpicture}
    \caption{Caption of the FSM}
    \label{fig:my_label}
\end{figure}

\subsection{Collision Avoidance with Random Walk}
