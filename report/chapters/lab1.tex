%!TEX root = ../main.tex
\section{Lab Activity 1}

First lab activity has been about phototaxis and an implementation of a
collision avoidance behaviour with a simple random walk.
As discussed in class, phototaxis has been more challenging because
it has been our first approach in programming robots in a more structured
way, and also the first time interacting with ARGoS.

\subsection{Phototaxis}\label{ssec:phototaxis}

Having no constraints on how fast the robot has to reach the light, and
no constraints in what the robot must do when reaching the light, designing
phototaxis is quite straightforward. Moreover, the arena is just composed
of its four walls, meaning that light cannot be obstructed by obstacles.

Having defined the environment and the constraints, the robot is designed through
a finite state automaton as illustrated in \Cref{fig:photo-fsa}.

The robot behaviour can be summarised in the following way: at each step,
locate the sensor with the highest value, extract its angle and determine if it
is on front, on the right or on the left. The first check sets the same
velocity for both wheels if the sensor with the highest value is the first or
the twenty-fourth. If this check returns false, sensor's angle is used to
determine if it is located on the left or right; let's call the angle $\alpha$,
we determine the direction of turn as:
%
\begin{equation}
    \begin{cases}
        \text{left}, & \text{if } \alpha \in [0, \pi] \\
        \text{right}, & \text{if } \alpha \in [-\pi, 0) \\
    \end{cases}
\end{equation}
%
If direction is right, we decrease right wheel velocity by a constant at each
step, until 1, while increasing left wheel velocity until max speed is reached.
The opposite happens when the direction is left.

Due to the fact that no obstacles are present in the arena, and the task is
only to reach the nearest light, wheels' velocities are always $> 0$, making
turn angles wider, but making it possibile to move faster.

\begin{figure}[ht]
    \centering
    \begin{tikzpicture}
        \node[state, initial] (init) {init};
        \node[state, right=of init] (search) {Search Light};
        \node[state, above right=of search] (forward) {Move Forward};
        \node[state, below right=of search] (turn) {Turn};
        \node[state, right=of turn] (right) {Turn Right};
        \node[state, right=of forward] (left) {Turn Left};

        % Define transitions
        \draw[->] (init) -- node[midway, above] {Random Velocities} (search);
        \draw[->] (search) -- node[midway, above] {Light Directly Ahead} (forward);
        \draw[->] (search) -- node[pos=0.3, midway, below] {Light Not Directly Ahead} (turn);
        \draw[->] (turn) -- node[midway, below] {Angle $\geq$ 0} (right);
        \draw[->] (turn) -- node[midway, above] {Angle $<$ 0} (left);

        \draw[->] (right) .. controls +(up:2) and +(right:2) .. 
            node[pos=0.2, right] {\texttt{l++}, \texttt{r--}} (search);
        \draw[->] (left) .. controls +(up:2) and +(right:2) .. 
            node[pos=0.1, left] {\texttt{l--}, \texttt{r++}} (search);
        \draw[->] (forward) .. controls +(up:3) and +(up:3) .. 
            node[pos=0.2, left] {Set Max Speed} (search);
    \end{tikzpicture}
    \caption{Phototaxis FSA}
    \label{fig:photo-fsa}
\end{figure}

In order to make robot decisions more robust in presence of sensor noise, the
actual max value of the sensor is determined with a weighted average based on
neighbours. In fact, a high value on a sensor could likely mean high values on
its neighbours. With this idea in mind, a function
\texttt{neighbour\_weighted\_avg(idx, n, sensors)} is created as a utility
function (which be later used by almost all lab activities). It takes as input
a starting index (the sensor index we are currently considering), $n$
representing the number of left and right neighbours to consider, and a set of
sensors (e.g. proximity, light, ...).

\subsection{Collision Avoidance with Random Walk}
Starting from ideas illustrated in \Cref{ssec:phototaxis}, we define two major
sub-behaviours: random walking and collision avoidance.

\paragraph{Random Walk} this behaviour is implemented as illustrated by the
Professor in his first example. We count each simulation step, and if we reach
a certain number of steps, random left and right velocities are applied to
wheels.

\paragraph{Collision Avoidance} this behaviour is strongly inspired by
phototaxis, meaning that we extract the maximum perceived value from proximity
sensors, and if it is $>$ major than a threshold, an obstacle is detected. In
order to get the maximum perceived value, we use the already defined
\texttt{get\_max\_for\_sensor(idx, n, sensors)}. As a threshold, proximity
sensors should be able to perceive up to 10cm. With experimental evaluation, a
good threshold has been found at 0.55, letting the footbot avoid obstacles in a
very smooth way while keeping a great security distance (much more important
with movable obstacles like other robots). As per phototaxis, we then extract
the angle of the sensor that has perceived the maximum value, making it
possibile to understand if obstacle is detected on the left or right. If it is
detected on the left ($\alpha > 0$), left wheel velocity should be set to
maximum value, while setting to zero right wheel velocity, in order to make a
turn which has as pivot the right wheel keeping turning radius as short as
possibile. The opposite is applied if obstacle is detected on the right
($\alpha < 0$).

\bigskip
\noindent
Having defined the two separate behaviours, they are managed at each step in
the following way: at each step, we call the \texttt{avoid\_obstacles()}
function, which will return left and right wheel velocities if obstacles are
detected, \texttt{nil} otherwise. If obstacle is detected, proposed left and
right velocities are applied to wheels, ending current step, otherwise the
random walk procedure is called, making it handle eventual right and left wheel
velocities setting.

\begin{figure}[ht]
    \centering
    \begin{tikzpicture}
        \node[state, initial] (init) {Initialize};
        \node[state, right=of init] (check) {Check Proximity};
        \node[state, above right=of check] (avoid) {Avoid Obstacle};
        \node[state, above=of avoid] (turnright) {Turn Right};
        \node[state, below=of avoid] (turnleft) {Turn Left};
        \node[state, below right=of check] (walk) {Random Walk};
        \node[state, right=of walk] (update) {Update Step Count};
        \node[state, below=of update] (newvelocity) {New Velocities};
        
        % Define transitions
        \draw[->] (init) -- node[midway, above] {Random Velocities} (check);
        
        \draw[->] (check) -- node[midway, above] {Proximity $>$ Threshold} (avoid);
        \draw[->] (check) -- node[midway, below] {Proximity $\leq$ Threshold} (walk);
        
        \draw[->] (avoid) -- node[midway, left] {Angle $>$ 0} (turnright);
        \draw[->] (avoid) -- node[midway, left] {Angle $\leq$ 0} (turnleft);
        
        \draw[->] (turnright) .. controls +(right:3) and +(up:2) .. 
            node[pos=0.3, right] {$v_l = MAX\_VELOCITY$, $v_r = 0$} (check);
        \draw[->] (turnleft) .. controls +(right:3) and +(down:2) .. 
            node[pos=0.3, right] {$v_l = 0$, $v_r = MAX\_VELOCITY$} (check);
            
        \draw[->] (walk) -- node[midway, right] {Increment Step Count} (update);
        \draw[->] (update) -- node[midway, right] {Step Count \% MOVE\_STEPS $\neq$ 0} (check);
        \draw[->] (update) -- node[midway, left] {Step Count \% MOVE\_STEPS = 0} (newvelocity);
        \draw[->] (newvelocity) .. controls +(left:3) and +(down:2) .. 
            node[pos=0.3, left] {Random $v_l$, Random $v_r$} (check);
    \end{tikzpicture}
    \caption{Random Walk with Collision Avoidance FSA}
    \label{fig:randomwalk-collision-fsa}
\end{figure}
